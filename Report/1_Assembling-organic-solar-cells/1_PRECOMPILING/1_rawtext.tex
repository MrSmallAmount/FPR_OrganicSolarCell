
\section*{Introduction}%prevent numbering for the intro
\addcontentsline{toc}{section}{\protect\numberline{}\hspace{0.3cm}Introduction}%add intro to contents

The first viable \textit{inorganic} solar cells came to light at Bell~Laboratories in 1954 \cite{siliconSC_1}\cite{siliconSC_2} with power conversion efficiencies of about 6 percent. Though the commercial success was limited because of high production costs, Chapin, Pearson and Fuller proved to us that it is possible to harness significant amounts of solar energy for practical usage and thus set foot for photovoltaic advancements.\mypar
Today's challenge is however to build low-cost solar cells that can be tailored to their specific application and can be produced in large amounts with established manufacturing processes\footnote{For an example of a realized roll-to-roll production see \cite{rolltoroll}.}. This has lead us to \textbf{organic} solar cells (\OSC).\mypar
Because of low efficiencies of heterojunction solar cells developed in the 1980s \cite{tang}, new concepts arose like the >>bulk~heterojunction<< (\BHJ) solar cell \cite{heterojunk}. It promised to bypass the problem of the short diffusion length of excitons created in the heterojunction as well as the limited thickness of the junction layers in a heterojunction cell.\mypar
Bulk heterojunction solar cells (\BHSC) still stand as state of the art technology as they are subject of current research (see~\cite{modernbulkhetero}). In this lab course we went through the process of assembling 5 different sets of \BHSC\ and characterized them with the {\os\sefo AM 1.5} global reference spectrum. In the following we will review the assembly\footnote{>>Assembly<< because we did not perform the preparation of the used materials.} and evaluate the obtained $I$-$U$-Curves.

\section{Assembly of bulk heterojunction solar cells}\label{sec:assembly}
For each of the 5 sets of \BHSC's we performed all steps of the assembly on a square glass substrate with sides of approximately 2~cm. The substrates used are coated with two \ITO\ stripes (see \cite{labdesc} Figure~12) with a width of
\begin{equation*}
w = (1600\pm50)\;\text{\textmu m}.
\end{equation*}
Where we assumed a measurement uncertainty of an analog caliper. The viewing-angle-dependent visibility of the transparent \ITO\ stripes, created the necessity of a shallow scratch on the glass. This helped identifying the orientation of the substrate later on.\mypar



