
\section{Discussion}\label{sec:discussion}
The theoretical model for a photo generated current with a single diode, taking serial $R_s$ and shunt $R_{sh}$ resistances into account predicts the current to have the following implicit shape.\cite{source10}

\begin{align}
I\left(1+\frac{R_s}{R_{sh}}\right)-\frac{U}{R_{sh}} = I_s\exp\left(\frac{e(U-I R_s)}{n k_B T}-1\right)-I_{ph}
\end{align}

Here the photogenerated current is $I_{ph}$ and the saturation current under reverse bias $I_s$, as in the current at which point the current is invariant to an increase of the voltage. $e$ is the elementary charge, $k_B$ the Boltzmann-constant, $T$ the temperature and $n$ is the ideality factor describing how well the cell can be described with the ideal diode equation, being 1 if it's ideal. The shunt resistance relates to parasitic currents moving directly between the electrodes\cite{source11}.
\\\textbf{more here (?)}\\
The first set $\mathbb{S}_1$ is the only set for which if the intensity of the light irradiating the first set $\mathbb{S}_1$ decreases, so does its fill factor. Such a behavior could be due to a higher level of impurities in the cell, as it was exposed to air at room conditions for several days.


