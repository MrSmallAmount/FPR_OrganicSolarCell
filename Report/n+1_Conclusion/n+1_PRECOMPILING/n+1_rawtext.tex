
\section{Conclusion}
While preparing and conducting the experiment we learned a lot about the modern ways of electron microscopes and their detectors. We learned that the Inlens detector is a phenomenal tool for imaging nanostructures like the frustums we encountered in this experiment. With the captured images from the detector we were able to extract the mean height of the frustums to be $51(8)\;\mathrm{nm}$. This value with its corresponding uncertainty lies outside of the range of the given value of 60 nm in \cite{labdesc} where we suggested to use a numeric algorithm that will manage the acquisition of the data necessary for the determination of the height.\mypar
In conclusion the GeminiSEM 500 represent state of the art technology to analyze specimen regrading their topological relief, their chemical composition and spatially resolved composition.

